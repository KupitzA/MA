\pagestyle{headings}
\pagenumbering{arabic}
\chapter{Introduction}
\label{chapter:intro}
When the expression of genes is studied, DNA methylation is a major issue which should be considered. This DNA modification mainly occurs at the 5 position of the pyrimidine ring of a cytosine base in the DNA\cite{DNAMethylation}. The methylated cytosine is thus often called \ac{5mC}. Whereas \ac{5mC} is rather rare over the whole genome, there exist small regions with high methylation frequency, so called \acp{CGI}. Variations in the methylation pattern of \acp{CGI} are related to changes in gene expression and cancer\cite{Handbook}. Additionally, methylcytosine  is associated to the process of genomic imprinting and X-chromosome inactivation\cite{Walter}.\newline
To study the transition of methylations from one cell generation to another, one needs to focus on \acfp{DNMT}. The general conformation is that there exist two kinds of methyltransferases, which may be discriminated by their function. DNMT1 seems to copy methylations from the parent to the daughter strand and its activity is related to the replication machinery. While DNMT3 may work at different positions in the DNA without any environmental conditions\cite{DNAMethylation}. In order to deepen the knowledge of DNA methylations and their spreading, computer methods are used to simulate their behaviour. More precisely, the methylation rates and their dependence on other system parameters are estimated. Therefore, it is made use of the fact that there is knowledge about the methylation state of specific loci from biological experiments.\\

In \cite{Genereaux} developed one of the first models for \ac{DNMT} simulation. In \cite{errors} and \cite{Wolf} the general idea is extended, allowing the possibility of errors in the data. 2016 in \cite{Giehr} a model was published that includes demethylation events during the replication process. But all these models do not include any location-dependency of the methylation rates despite the data allows the assumption.\cite{errors}\newline
This idea of neighbourhood- and location-dependency was regarded in multiple approaches. In \cite{Fu}, paper from 2012, the processivity of \acp{DNMT} was investigated, availing the methylation states of all preceding positions and trying to infer the binding state of the enzymes. However, the approach fails to identify all parameters.\newline
Integrating the limitations of the previous method, Bonello et al.\cite{Bonello} compare different location- and neighbour-dependent models to spot the model which seems to fit best to the real-world data. They conclude, that a model which respects the neighbouring positions is most likely to result in the desired data.\newline
Recently, a aimilar approach\cite{Lueck} was published that takes direct neighbouring methylation states into account. The realized experiments reason a dependence on the left-, but not on the rightmost neighbour. Nevertheless, the binding state of the \acp{DNMT} is not kept track of and thus the association of the enzyme to the replication process is still unknown.\\

Here we will present an approach which is based on the same parameters for \acp{DNMT} as in the paper by Fu et al.(2012)\cite{Fu}. The related works use different models and methods to estimate the process parameters. We simulate the transmission of methylations in one way, but compare the computational methods for parameter estimation. We show that we are able to infer the function of the two kinds of methyltransferases and that their their methylation rates are differentiable. These results agree to common, recent findings and suppositions, which we are able to prove.\\
TODO: extend... (why is this model useful)\\